\section{Metamaterial absorbers}
Metamaterials are artificial materials with customized electromagnetic properties. 
They can, among a vast number of other applications, be used to design flat absorbers for electromagetnic waves which can efficiently reduce the radar cross section (RCS) of given objects. Low RCS values are desireable in situations where unwanted reflections in radiating systems like antennas are to be avoided or, in case of military vehicles, to achieve radar-invisibility.

In order to reduce radar backscattering either the geometry or used materials can be optimized within a defined scope. Of course, a RCS reduction mechanism which does only change the electromagnetic scattering properties but does not alter the shape of a given object too much would be most desireable from an aerodynamical or mechanical point of view. Metamaterial absorbers can beat their conventional competitors, absorbing foams, when it comes to very flat designs, since the latter usually possess thicknesses of $\lambda/2$, where $\lambda$ is the largest frequency which can be absorbed by the absorber, whereas metamaterials can do a good job with thicknesses in the range of $\lambda/100$ to $ \lambda/10$, with higher values for a larger freqeuncy bandwidth.

Due to their working principle deflectors and absorbers can be discriminated. The former include "chessboard-like" structures which scatter normally incident waves into side-lobes and prevent mirror-like backscattering. The latter include periodic structures which are resonant at one or more frequencies and can trap and absorb incoming radiation. Typical primitive building blocks can be Jerusalem crosses, "gangbusters", rectangles, split-rings and many more \cite{Munk2000}.

The primitive building block of these periodic structures is called unit cell. In section \ref{sec:double_rings} a certain type of unit cell with rotational symmetry around the $z$-axis is proposed. It's main features are two resonant frequencies which can be adjusted independently by varying geometrical parameters. Due to the symmetry of the unit cell the resulting scattering properties are not very sensitive to the polarization of impinging electromagnetic waves.