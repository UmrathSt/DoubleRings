\section{Metamaterial absorbers}
A metamaterial absorber is an artificial structure which is designed in order to
reduce the radar cross section (RCS) which is a measure of the detection probability upon illumination by electromagnetic waves. Possible areas of application 
may be the reduction of unwanted reflections in radiating systems like antennas, or backscattering reduction of military vehicles in order to achieve radar-invisibility.

In order to reduce radar backscattering either the geometry or used materials can be optimized within a defined scope. Of course, a RCS reduction mechanism which does only change the electromagnetic scattering properties but does not alter the shape of a given object too much would be most desireable from an aerodynamical or mechanical point of view.

Metamaterials are possible solutions for RCS reduction. Due to their working principle deflectors and absorbers can be discriminated. The former include "`chess-board"'-like structures which scatter normally incident waves into side-lobes and prevent mirror-like backscattering. The latter include periodic structures consisting of rings, split-rings, Jerusalem crosses and many more.

The primitive building block of these periodic structures is called unit cell. In section \ref{sec:double_rings} a certain type of unit cell with rotational symmetry around the $z$-axis is proposed. It's main features are two resonant frequencies which can be adjusted independently by varying geometrical parameters. Due to the symmetry of the unit cell the resulting scattering properties are not very sensitive to the polarization of impinging electromagnetic waves.