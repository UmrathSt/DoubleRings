\section{An analytical model of the scattering by rings}
\label{sec:analytical_ring_scattering}
A plane wave of frequency $\omega$ with the elctric field vector pointing along $\unitv{\xi}$ and propagating in the direction of $\mathbf{k}$ can be written: 
\begin{equation}
\mathbf{E}^\text{inc.} (\unitv{r}, t)= E_0 \unitv{\xi} \exp\left[\imag (-\omega t- \mathbf{k}\cdot \mathbf{r})\right].
\label{eqn:Einc_cart}
\end{equation}
In order to describe how a 2d ring scatters waves described by \cref{eqn:Einc_cart} an appropriate coordinate system should be chosen. This could be both, cylindrical ($\rho$, $\phi$, $z$) or spherical coordinates ($r$, $\theta$, $\phi$). From now on we will choose spherical coordinates and drop the harmonic time dependence. A complete set of zero-divergence (i. e. sourceless) vector fields is given by the Hansen multipoles \Cite{Hansen1935} 
\begin{equation}
\mathbf{M}_{\ell m} = \nabla \times \unitv{r} z_\ell(kr) Y_{\ell m}(\theta, \phi),
\label{eqn:HansenM}
\end{equation}
\begin{equation}
\mathbf{N}_{\ell m} = \frac{1}{k}\nabla \times \mathbf{M}_{\ell m}.
\label{eqn:HansenN}
\end{equation}
In \cref{eqn:HansenM} $z_\ell$ is any linear combination of spherical bessel functions $j_\ell(kr)$ and $y_\ell(kr)$ \Cite{DLMF_Bessel} of degree $\ell$. $r$ is the distance from the origin and $k$ the wave number. $Y_{\ell m}$ denote spherical harmonics of degree $\ell$ and order $m$ \Cite{DLMF_SphericalHarmonics}. Their dependence upon the azimut angle $\theta$ and polar angle $\phi$ is understood although not written out explicitly from now on.

Written in terms of $\mathbf{M}_{\ell m}$ and $\mathbf{N}_{\ell m}$ \cref{eqn:Einc_cart} becomes:
\begin{equation}
\unitv{\xi} \exp\left[-\imag  \mathbf{k}\cdot \mathbf{r}\right] =\sum_{\ell,m} \sum_{m'=-\ell}^\ell\mathcal{D}_{m m'}^\ell (\theta_k, \phi_k,0)\big[
m_{\ell, m} \mathbf{M}_{\ell m} +n_{\ell,m} \mathbf{N}_{\ell m}\big].
\end{equation}
Here $\theta_k$ and $\phi_k$ denote the angular coordinates of $\mathbf{k}$ and $\mathcal{D}_{m m'}^\ell$ are Wigner-D matrices in $zxz$ convention. The coefficients $m$ and $n$ can be evaluated to \Cite{Kristensson2014}:
\begin{align}
&& m_{\ell,m} = &\imag^{\ell+1} \delta_{m\pm 1}\sqrt{(2\ell+1)\pi} 
(\unitv{x}\mp\imag \unitv{y})\cdot \unitv{\xi},\\
&& n_{\ell,m} = \pm&\imag^{\ell+1} \delta_{m\pm 1}\sqrt{(2\ell+1)\pi} 
(\unitv{x}\mp\imag \unitv{y})\cdot \unitv{\xi}.
\end{align}

An electromagnetic wave incident on a conducting ring (radius $R$ and width $w$) situated in the $xy$-plane induces currents in the ring which can only exist on the surface of the rings. If the ring possesses zero thickness in $z$-direction and zero width $w$ only currents flowing in $\unitv{\phi}$-direction can exist. Therefore we may expand the current density responsible for scattered waves in a fourier series in the coordinate $\phi$:
\begin{equation}
\mathbf{j}^\text{scat} = \frac{1}{2\pi R}\sum_{p=-\infty}^\infty \mathrm{e}^{\imag p \phi} \delta(r-R) \delta(\theta-\pi/2) \unitv{\phi}K_p.
\label{eqn:ring_currents}
\end{equation}
Note that \cref{eqn:ring_currents} needs to be extended to include $\unitv{\theta}$ and $\unitv{r}$ components, if the ring possesses a finite thickness or a finite width, respectively. PEC boundary conditions force the components of $\mathbf{E}$ tangential to the surface of the ring to zero and by that determine the coefficients $K_p$.

The fields generated by the current density of \cref{eqn:ring_currents} can be
calculated by using the dyadic Green's function written in terms of vector spherical harmonics\footnote{There is also a freely available version of Gerhard Kristensson's lecture notes on "Spherical Vector Waves" \citep{Kristensson2014}.} \Cite{Kristensson2016}[chapter 7.5]:
\begin{equation}
\dyad{\mathbf{G}}(\mathbf{r},\mathbf{r}') = \imag k \sum_{\ell=1}^\infty \sum_{m=-\ell}^\ell
\mathbf{M}_{\ell m}(\mathbf{r})\otimes\mathbf{M}_{\ell m}^*(\mathbf{r}')+
\mathbf{N}_{\ell m}(\mathbf{r})\otimes\mathbf{N}_{\ell m}^*(\mathbf{r}'),
\label{eqn:Greens_Dyadic}
\end{equation}
where the vectorfield depending on the position vector with the smaller magnitude, $r_<=\min\left(r, r'\right)$, is to be understood to depend on regular spherical bessel functions $j_\ell(kr_<)$ whereas the other vectorfield possesses a dependence on the radial coordinate $r_>=\max\left(r, r'\right)$ via outgoing\footnote{The term outgoing is owing to the choice of the $\exp\left(-\imag\omega t\right)$ time dependence.} spherical hankel functions of the first type, $h_\ell^{(1)}(kr_>)$.
The asterix $^*$ means complex conjugation. With the aid of \cref{eqn:Greens_Dyadic} the scattered electric field can be calculated via:
\begin{equation}
\mathbf{E}^\text{scat}(\mathbf{r}) = \int_{\mathcal{R}^3}  \rd \mathbf{r}' 
\dyad{\mathbf{G}}(\mathbf{r},\mathbf{r}') \cdot \mathbf{j}^\text{scat}(\mathbf{r}').
\label{eqn:E_scattered}
\end{equation}

The scattered field of \cref{eqn:E_scattered} can be therefore cast in the form:
\begin{equation}
\mathbf{E}^\text{scat}(\mathbf{r}) = \imag k \sum_{\ell=1}^\infty  \sum_{m=-\ell}^\ell
I_{\ell m}^\textbf{(M)}\mathbf{M}_{\ell m}(\mathbf{r})+
I_{\ell m}^\textbf{(N)}\mathbf{N}_{\ell m}(\mathbf{r})
\label{eqn:Escat}
\end{equation}
with the expansion coefficients
\begin{align}
I_{\ell m}^\text{(M)} = \int \rd^3\mathbf{r}' \mathbf{M}^*_{\ell m}(\mathbf{r}')\cdot
\mathbf{j}^\textbf{scat}(\mathbf{r}'),\\
I_{\ell m}^\text{(N)} = \int \rd^3\mathbf{r}' \mathbf{N}^*_{\ell m}(\mathbf{r}')\cdot
\mathbf{j}^\textbf{scat}(\mathbf{r}'),
\end{align}
where the Hansen multipoles have to be written with regular spherical bessel function $j_\ell(kr)$. The above integrals can be evaluated in closed form, if \cref{eqn:ring_currents} is inserted for the scattered current density $\mathbf{j}^\text{scat}$ yielding:
\begin{align}
\left.I_{\ell m}^\textbf{(M)} = \delta_{p m}\frac{j_\ell(kR)}{\sqrt{\ell(\ell+1)}} \left[-K_p\frac{\displaystyle\rd}{\rd \theta}P_{\ell}^{(m)}(\cos(\theta))
\right]\right \vert_{\theta=\pi/2},\\
\left.I_{\ell m}^\textbf{(N)} =  \delta_{p m} 
\frac{\frac{\rd}{\rd (kR)}\left[kR j_\ell(kR)\right]}{kR\sqrt{\ell(\ell+1)}}\left[-\imag m K_p
P_\ell^{(m)}(\cos(\theta)) \right]\right\vert_{\theta=\pi/2}.
\end{align}
The associated Legendre polynomials and their derivative can be written \Cite{Leong1997}:
\begin{align}
\frac{\rd}{\rd \theta}P_\ell^{(m)}(0) =\left\{\ontop{}{}{
-\sin\big((\ell+m)\pi/2\big)\dfrac{2^{m+1}\Gamma\big((l+m)/2+1\big)}{\sqrt{\pi}\Gamma\big((l-m+1)/2\big) }
\quad : \quad (l+m)\%2=1}{\quad0\quad: \quad\text{otherwise}}\right\},
\label{eqn:Plm0diff}
\\\left\{
P_\ell^{(m)}(0) = \ontop{}{}{\cos\big((l+m)\pi/2\big)\dfrac{2^m \Gamma\big((n+m+1)/2\big)}{\sqrt{\pi}\Gamma\big((n-m)/2+1\big)}\quad:\quad (l+m)\%2=0}{0\quad:\quad \text{otherwise}}\right\}.
\label{eqn:Plm0}
\end{align}
If we are just interested in the scattered field and not the current distribution in the rings, we find the $I$-coefficients by requiring that for each occuring term proportional to $\mathbf{M}_{\ell,m}$ and $\mathbf{N}_{\ell,m}$ in the expansion of the incoming field the respective terms on the right-hand side of \cref{eqn:E_scattered} must ensure that 
\begin{align}
\left(
E_0^\text{inc}\left[
\begin{matrix}
\mathbf{M}_{\ell,m}^\text{reg} \\
\mathbf{N}_{\ell,m}^\text{reg}
\end{matrix}\right]
+
\imag k \left[\begin{matrix}
I_{\ell m}^\textbf{(M)}\mathbf{M}_{\ell,m}^\text{out} \\
I_{\ell m}^\textbf{(N)}\mathbf{N}_{\ell,m}^\text{out}
\end{matrix}\right]\right) \cdot 
\unitv{\phi}\left.\right\vert_{\theta=\pi/2,r=R} = 0
\end{align}

\subsection{Perpendicular incidence, perfectly conducting rings}
At perpendicular incidence the cylindrical symmetry of the rings allows to choose the polarization in $\unitv{x}$-direction without loss of generality. Therefore \cref{eqn:Einc} becomes:
\begin{equation}
\mathbf{E}^\text{inc.} (z, t)= \unitv{x} \exp\left[\imag (-\omega t- k\cdot z)\right].
\label{eqn:Ex}
\end{equation}
Henceforth the time-dependence $\exp(-\imag \omega t)$ will be suppressed. To satisfy the boundary conditions
\begin{align}
\left.\left(\mathbf{E}^\text{inc.}+\mathbf{E}^\text{scat}\right)\times \unitv{r}\right\vert_{\theta=\pi/2, r=R} = 0,
\label{eqn:Etan}\\
\left.\left(\mathbf{H}^\text{inc.}+\mathbf{H}^\text{scat}\right)\times \unitv{r}\right\vert_{\theta=\pi/2, r=R} = 0,
\label{eqn:Htan}
\end{align}
for the field components transverse to the rings we expand \cref{eqn:Ex} in Hansen multipoles \cite{Kristensson2014}[p. 44 ff.]:
\begin{equation}
\mathbf{E}^\text{inc.} = \sqrt{(2\ell+1)\pi}\sum_{\ell=1}^\infty \imag^{\ell+1}\ 
\big( \mathbf{M}_{\ell m} \pm \imag \mathbf{N}_{\ell m}\big) \delta_{m \pm 1}
\label{eqn:Einc}
\end{equation}
and have to take care that the $\unitv{\phi}$- and $\unitv{\theta}$-components of the total fields satisfy \cref{eqn:Etan,eqn:Htan}. Supposing for the moment, that the ring is small as compared to the wavelength, we can restrict the sums in \cref{eqn:Escat} to $\ell=1$. This assumption leads to  $I^\text{(M)}_{1,\pm 1}=0$ (see \cref{eqn:Plm0diff}), whereas $I^\text{(N)}_{1,\pm 1}$ (see \cref{eqn:Plm0}) survives for $|m|=1$.
\begin{align}
\left[
\renewcommand\arraystretch{1.4}\begin{matrix}
I_{1,1}^\text{(M)} \\
I_{1,0}^\text{(M)} \\ 
I_{1,-1}^\text{(M)}
\end{matrix} \right]=
\frac{j_1(kR)}{\sqrt{2}} \left[ 
\renewcommand\arraystretch{1.4}\begin{matrix}
0 \\ K_0 \\ 0
\end{matrix}
\right]
\\
\left[
\renewcommand\arraystretch{1.4}\begin{matrix}
I_{1,1}^\text{(N)} \\
I_{1,0}^\text{(N)} \\ 
I_{1,-1}^\text{(N)}
\end{matrix} \right] =  \frac{\frac{\rd}{\rd (kR)}\left[kR j_1(kR)\right]}{kR\sqrt{2}} 
\left[
\renewcommand\arraystretch{1.4}\begin{matrix}
-K_1\\0\\+K_{-1}
\end{matrix}
\right]
\end{align}

We observe, that in order to match the boundary conditions the coefficients $I_{\ell m}^\text{(M)}$ and $I_{\ell m}^\text{(N)}$ resemble Mie's coefficients for a perfectly conducting sphere of radius $R$, $b_\ell$ and $a_\ell$. 
The fact that on the thin ring no currents in the $\unitv{\theta}$-direction can be induced leads to the vanishing magnetic dipole polarizability in the $xy$-plane i. e. zero terms $I_{\ell, \pm 1}^\text{(M)}$ and a vanishing electric dipole polarizability in the $\unitv{z}$-direction $I_{\ell, 0}^\text{(N)}$.

\subsection{xy-periodic array of thin rings}
Building upon the results of the last section we can discuss the scattering of a periodic arrangement of rings in the $xy$-plane by forcing the expansion coefficients $Q^\text{TM}$ and $Q^\text{TE}$ of the total electric field to be eigenfunctions of the translationoperators $\mathcal{P}_x$ and $\mathcal{P}_y$. This is necessary because the fields in the periodic structure must satisfy:
\begin{equation}
\mathbf{E}\left(\mathbf{r}+m L \unitv{x}+
n L \unitv{y}\right)=
\mathbf{E}\left(\mathbf{r}\right).
\end{equation}
For any choice of the whole numbers $m, n\in \mathbb{Z}$. Therefore we can write
\begin{equation}
\mathbf{E}\left(\mathbf{r}\right) = 
\exp\left(\imag \frac{2\pi}{L}x\right) \exp\left(\imag \frac{2\pi}{L}y\right)\mathbf{\tilde{E}}\left(\mathbf{r}\right),
\end{equation}
where $\mathbf{\tilde{E}}(\mathbf{r})$ is the field in a single unit-cell. Each of the exponential factors can be expanded in the fashion:
\begin{equation}
\exp\left(\imag\mathbf{k}\cdot \mathbf{r}\right) = 4\pi \sum_{\ell=0}^\infty \sum_{m=-\ell}^\ell
\imag^\ell j_\ell(kr) Y_{\ell m}(\unitv{r}) Y_{\ell m}^*(\unitv{k}),
\end{equation}
where the arguments $\unitv{r}$, $\unitv{k}$ of the spherical harmonics are a shortcut for the dependence upon the angles $\theta$, $\phi$ of the position vector $\mathbf{r}$ and $\theta_\mathrm{k}$, $\phi_\mathrm{k}$ of the wave vector $\mathbf{k}$. Substituting $\mathbf{k}=\frac{2\pi}{L}\unitv{x}$ we end up with:
\begin{align}
&& \nonumber
\exp\left(\imag \frac{2\pi}{L} x\right) &= 4\pi \sum_{\ell=0}^\infty \sum_{m=-\ell}^\ell
\imag^\ell j_\ell\left(\frac{2\pi r}{L}\right) Y_{\ell m}(\theta, \phi) Y_{\ell m}^*\left(\frac{\pi}{2}, 0\right) \\
&& &= 4\pi N_{\ell m} \sum_{\ell=0}^\infty \sum_{m=-\ell}^\ell
\imag^\ell j_\ell\left(\frac{2\pi r}{L}\right) Y_{\ell m}(\theta, \phi)
P_{\ell m}(0).
\label{eqn:xperiodicity}
\end{align}
This sum contains only terms of pairs $(\ell,m)$ with even $\ell+m$ (see \cref{eqn:Plm0}). If we collect both, the exponential depending on $x$ and $y$ we can combine them to:
\begin{equation}
\exp\left(\imag \frac{2\pi}{L} (x+y)\right) = 4\pi \sum_{\ell=0}^\infty \sum_{m=-\ell}^\ell
\imag^\ell j_\ell\left(\frac{2\pi r}{L}\right) Y_{\ell m}(\theta, \phi) Y_{\ell m}^*\left(\frac{\pi}{2}, \pi/4\right).
\end{equation}

In total we end up with the boundary condition:
\begin{equation}
\left[\mathbf{E}^\text{inc}+\exp\left(\imag \frac{2\pi}{L}(x+y)\right)\mathbf{E}^\text{scat}\right] \cdot \unitv{\phi}\big|_{r=R} = 0
\label{eqn:xyperiodicBC}
\end{equation}

\subsubsection{Dipole limit}
If the radius of the ring is small compared to the wavelength of the incident radiation, the $\ell=1$-contribution to the scattered field is dominant.
In this case \cref{eqn:xyperiodicBC} becomes:
\begin{equation}
m_{1,m} \mathbf{M}_{1 m} + n_{1 m} \mathbf{N}_{1 m} = \sum_{\ell'=0}^\infty\sum_{m'=-\ell}^{\ell'}
C_{\ell' m'} Y_{\ell' m'} 
\left(
m^\text{scat}_{1 m''} \mathbf{M}_{\ell'' m''} +
n^\text{scat}_{1 m''} \mathbf{N}_{\ell'' m''}
\right)
\end{equation}