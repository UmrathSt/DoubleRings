\section{Metamaterial absorbers}
Metamaterials are artificial materials with customized electromagnetic properties. 
They can, among a vast number of other applications, be used to design flat absorbers for electromagetnic waves which can efficiently reduce the radar cross section (RCS) of given objects. Low RCS values are desireable in situations where unwanted reflections in radiating systems like antennas are to be avoided or, in case of military vehicles, to achieve radar-invisibility.

In order to reduce radar backscattering either the geometry or the used materials can be optimized within a defined scope. Of course, a RCS reduction mechanism which does only change the electromagnetic scattering properties but does not alter the shape of a given object too much would be most desireable from an aerodynamical or mechanical point of view. Metamaterial absorbers can beat their conventional competitors, absorbing foams, when it comes to very flat designs, since the latter usually possess thicknesses of $\lambda/2$, where $\lambda$ is the largest frequency which can be absorbed by the absorber, whereas metamaterials can do a good job with thicknesses in the range of $\lambda/100$ to $ \lambda/10$, with higher values for a larger freqeuncy bandwidth.

Due to their working principle deflectors and absorbers can be discriminated. The former include "chessboard-like" structures which scatter normally incident waves into side-lobes and prevent mirror-like backscattering. The latter include periodic structures which are resonant at one or more frequencies and can trap and absorb incoming radiation. Typical primitive building blocks can be Jerusalem crosses, "gangbusters", rectangles, split-rings and many more \cite{Munk2000}.

The primitive building block of these periodic structures is called unit cell. Typically a metamaterial absorber consistis of copper structures upon one or more layers of some kind of dielectric substrate. Since the dielectric substrates are isotropic in the xy-plane, their scattering properties can be treated seperately from the meta-sheet in order to optimize their thicknesses.

\subsection{Some definitions}

If we have a structure with three different media like depicted in \cref{fig:stacked_structure}, we can describe the scattering of the structure like:
\begin{equation}
\Gamma_1 = \frac{\rho_1+\rho_2 \exp\left[-2\imag k_2 L_2\right]}
{1+\rho_1\rho_2 \exp\left[-2\imag k_2 L_2\right]},
\label{eqn:2layerGamma}
\end{equation}
where the the coefficients $\rho$ are indexed by the number of the interface and 
\begin{equation}
\rho_j = \frac{Z_j - Z_{j-1} }{Z_j + Z_{j-1} }
\label{eqn:rho}
\end{equation}
stands for the scattering factor at interface $j$ and depends on the impedances left and right to the interface, 
$Z_{j-1}$ and $Z_j$. The length of the layer under study is $L_2$ and the wavenumber in terms of permittivity 
and permeability is $k_2=\sqrt{\varepsilon_2\mu_2} \omega/c$, whereas $Z_2=\sqrt{\mu_2/\varepsilon_2}$ and permittivites and permeabilities are to be understood as relative quantitites.


\begin{figure}
\centering
\begin{tikzpicture}
\tikz \filldraw[fill=rgray3,draw=black] (0,0) rectangle (3,3) ++(-1.5,0.25) node[black] {$Z_1$};
\tikz \filldraw[fill=rgray1,draw=black] (3,0) rectangle (4,3) ++(-0.5,0.25) node[black] {$Z_2$};
\tikz \filldraw[fill=rgray3,draw=black] (4,0) rectangle (7,3) ++(-1.5,0.25) node[black] {$Z_3$};
\end{tikzpicture}
\caption{The scattering properties of a metamaterial described by an impedance $Z_2$ which is embedded (from the left) in a material of $Z_1$ and from the right in $Z_3$.}
\label{fig:stacked_structure}
\end{figure}

\cref{eqn:2layerGamma} can be generalized for structures with more than one layer and in general becomes:
\begin{equation}
\Gamma_{j-1} = \frac{\rho_{j-1}+\Gamma_j \exp\left[-2\imag k_j L_j\right]}
{1+\rho_{j-1}\Gamma_j \exp\left[-2\imag k_j L_j\right]}.
\label{eqn:NlayerGamma}
\end{equation}

In the case of metamaterial absorbers there are often three layers. From vacuum with impedance $Z_0$ a wave is incident on a very thin sheet of some copper structure with impedance $Z_m$ followed by a substrate layer of $Z_s$ with thickness $l_z$ which is backed by a conducting surface with impedance $Z_g$, which in case of perfect conductivity is exactly zero.

\begin{figure}
\centering
\begin{tikzpicture}
\tikz \filldraw[fill=rgray3,draw=black] (0,0) rectangle (3,3) ++(-1.5,0.25) node[black] {$Z_0$};
\tikz \filldraw[fill=rgray1,draw=black] (3,0) rectangle (4,3) ++(-0.5,0.25) node[black] {$Z_m$};
\tikz \filldraw[fill=rgray1,draw=black] (3,0) rectangle (4,3) ++(-0.5,0.25) node[black] {$Z_s$};
\tikz \filldraw[fill=rgray3,draw=black] (4,0) rectangle (7,3) ++(-1.5,0.25) node[black] {$Z_g\to 0$};
\end{tikzpicture}
\caption{The scattering properties of a typical metamaterial absorber is described by an impedance $Z_m$ which is on top of a grounded substrate slab.}
\label{fig:Nstacked_structure}
\end{figure}

In this case the total scattering coefficient on the interface between $Z_0$ and $Z_m$ can be evaluated once the scattering coefficients at the interfaces $Z_m$/$Z_s$ and $Z_s/Z_g$ are known.

From \cref{eqn:rho} we get -- in case of $Z_g=0$ for the scattering at $Z_s/Z_g$ that $\Gamma_3=\rho_3=-1$. Therefore the scattering coefficient $\Gamma_2$ of the interface $Z_m/Z_s$ becomes:
\begin{equation}
\Gamma_2 = \frac{\rho_2- \exp[-2\imag k_2 L_2]}
			    {1-\rho_2\exp[-2\imag k_2 L_2]}
	    =
\end{equation}

\subsection{Metamaterial absorber layout}
A typical metamaterial absorber consists of a 2d patch with metallic structures on top of a dielectric substrate.
It can be modelled as a parallel connection of the impedance of a grounded dielectric substrate:
\begin{equation}
Z_d = \frac{jZ_0}{\sqrt{\epsilon_\mathrm{r}}} \tan\left(\omega/c_0 \sqrt{\epsilon_\mathrm{r}}d\right),
\end{equation}
and the impedance $Z_{fss}$ of the metallic patch layer / frequency selective surface.

\begin{figure}[h!]
  \begin{center}
    \begin{circuitikz}[european resistors]
      \draw (0,0) to[short,o-] (2,0) 
      to[R=$R$] (2,2) % The resistor
      to[L=$L$] (2,3) % The resistor
      to[C=$C$] (2,4) % The resistor
	  to[short,-o] (0,4)
      (2,0) to[short] (4,0)
      to[R=$Z_d$] (4,4)
      to[short] (2,4);
    \end{circuitikz}
    \caption{Transmission line model of a FSS patch on top of a grounded dielectric substrate with impedance $Z_d$.}
  \end{center}
\end{figure}

The resulting impedance $Z_m$ of the metamaterial becomes:
\begin{align}
\nonumber
&&\frac{1}{Z_\mathrm{m}} &= \frac{1}{Z_\mathrm{d}} + \frac{1}{Z_\mathrm{fss}}\\
&&\Rightarrow Z_\mathrm{m} &= \frac{Z_\mathrm{d} Z_\mathrm{fss}}{Z_d + Z_\mathrm{fss}}.
\end{equation}

While $Z_\mathrm{d}$ is given analytically, $Z_\mathrm{fss}$ must be calculated from simulations. Therefore the reflection coefficient
$\Gamma$ can be utilized which reads:
\begin{align}
\nonumber
&&\Gamma_\mathrm{fss} &= \frac{Z_\mathrm{fss}-Z_0}{Z_0 + Z_\mathrm{fss}}\\
&&\Rightarrow Z_\mathrm{fss} &= Z_0 \frac{1 - \Gamma_\mathrm{fss}}{1 + \Gamma_\mathrm{fss}}.
\end{align}