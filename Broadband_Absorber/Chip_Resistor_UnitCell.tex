\documentclass{article}

\usepackage{tikz}
\usetikzlibrary{arrows,calc,decorations.markings}
\usetikzlibrary{math}
\usepackage{xintexpr}
\usepackage{siunitx}
\usetikzlibrary{external}

\tikzexternalize[prefix=tikzfigures/]

\begin{document}
\pgfarrowsdeclarecombine{|<}{>|}{|}{|}{latex}{latex}
\def\Dimline[#1][#2][#3]{
    \begin{scope}[>=latex] % redef arrow for dimension lines
        \draw let \p1=#1, \p2=#2, \n0={veclen(\x2-\x1,\y2-\y1)} in [|<->|,
        decoration={markings, % switch on markings
                mark=at position .5 with {\node[#3] at (0,0) {\DimScale{\n0}};},
        },
        postaction=decorate] #1 -- #2 ;
    \end{scope}
}% 

\def\DimScale#1{\pgfmathparse{round(#1/28.33333333*10)/10} \pgfmathprintnumberto[precision=4]{\pgfmathresult}{\roundednumber}  {\roundednumber $\,$}mm}
\newcommand*{\pgfmathsetnewmacro}[2]{%
    \newcommand*{#1}{}% Error if already defined
    \pgfmathsetmacro{#1}{#2}%
}%


\def\UCDim{14}
\def\Length{4}
\def\sLength{3}
\def\fgap{0.6}
\def\sgap{0.6}
\pgfmathsetnewmacro{\ssLength}{2}
\pgfmathsetnewmacro{\fLength}{2*\Length+\fgap}
\def\reswidth{0.6}

%% The following macro is used to scale a dimension from points to the
%% display scale.  The following code divides the number of points by
%% 28.4 to roughly get the width in centimeters (rounding to the
%% nearest millimeter):



\tikzsetnextfilename{UnitCell}
\begin{tikzpicture}
\newcommand{\rect}{(\fLength/2-\sLength, \fgap/2) -- (\fLength/2, \fgap/2) -- 
				   (\fLength/2, \fgap/2+\Length)  -- (\fgap/2, \fgap/2+\Length) --
				   (\fgap/2, \fLength/2-\sLength)
				   };
	\fill[orange, opacity=0.3] \rect -- cycle;
	\Dimline[($(\fLength/2,\fgap/2)$)][($(\fLength/2,\fgap/2+\Length)$)][right];
	\Dimline[($(\fLength/2-\sLength, \fgap/2)$)][($(\fLength/2, \fgap/2)$)][above];
	\Dimline[($(-\fgap/2, \fgap/2+\Length)$)][($(\fgap/2, \fgap/2+\Length)$)][above];
	\foreach \i in {90, 180, 270}
	{
	\begin{scope}[rotate around={\i:(0,0)}]
	\fill[orange, opacity=0.3]
        \rect -- cycle;	
	\end{scope}
	}
	\draw[black] (-\UCDim/2, -\UCDim/2) -- (+\UCDim/2, -\UCDim/2) --
				 (+\UCDim/2, +\UCDim/2) -- (-\UCDim/2, +\UCDim/2) --
			     cycle;
	\Dimline[($(-6.99, -7)$)][($(7, -7)$)][below];			     
			     
%    \node at (0,0) (nA) {A};
%    \node at (3,0) (nB) {B};
%    \Dimline[($(nA)+(0,1)$)][($(nB)+(0,1)$)][above];
%
%    \node at (0,-3) (nC) {C};
%    \Dimline[($(nA)+(-1,0)$)][($(nC)+(-1,0)$)][left];
%    \Dimline[($(nC)+(0.3,-0.3)$)][($(nB)+(0.3,-0.3)$)][right];
%
%    \node at (3,-3) (nD) {D};
%    \Dimline[($(nC)+(0,-1)$)][($(nD)+(0,-1)$)][below];
%    \Dimline[($(nB)+(1,0)$)][($(nD)+(1,0)$)][right];
\end{tikzpicture}

\end{document}